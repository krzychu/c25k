\documentclass[parskip]{scrartcl}
\usepackage[margin=15mm]{geometry}
\usepackage{tikz}
\usepackage{pifont}
\usepackage{hyperref}

% global parameters
\pgfmathsetmacro{\cardwidth}{5}
\pgfmathsetmacro{\cardheight}{7}
\pgfmathsetmacro{\stripwidth}{0.7}
\pgfmathsetmacro{\strippadding}{0.1}
\pgfmathsetmacro{\textpadding}{0.3}
\pgfmathsetmacro{\ruleheight}{0.3}

\newcommand{\workoutcard}[3]{
  \begin{tikzpicture}
    % border
    \draw[rounded corners=0.2cm] (0,0) rectangle (\cardwidth,\cardheight);
   
    % week and day strip
    \fill[#1,rounded corners=0.1cm] (\strippadding,\strippadding) 
      rectangle (\strippadding+\stripwidth,\cardheight-\strippadding) 
      node[rotate=90,above left,black,font=\LARGE] {#2};
  
    % workout
    \node[text width=(\cardwidth-\strippadding-\stripwidth-2*\textpadding)*1cm,below right] 
      at (\strippadding+\stripwidth+\textpadding,\cardheight-\textpadding) 
    {
      \begin{center}
      \begin{tabular}{l|l}
      #3
      \end{tabular}
      \end{center}

    };
  \end{tikzpicture} 
}

\title{C25K card deck}

\begin{document}
\maketitle

\section{What is it?}
C25K (\url{http://www.coolrunning.com/engine/2/2\_3/181.shtml}) or
Couch -- to -- 5K is a popular training program for people who want
to start jogging. It aims to take you from absolute beginners level
to 5 kilometer nonstop run in just nine weeks. This document contains
deck of cards with workouts for every single day of training, so 
you don't have to memorize the intervals. Time on cards is
written in format \verb|mm:ss|.

\textbf{IMPORTANT: I'm not an author of this program. You must read
instruction from link above, and you train on your own responsibility}


\section{Who is it for?}
During interval training it's easy to lose track of what to do next 
when not having any "cheatsheet" with workout plan. Of course there are
plenty of C25K applications for smartphones, but if for some reason you
don't want to take yours for jogging, this card deck is for you.

I had one more reason for creating this document, because as board game
enthusiast I wanted to learn how to create deck of cards using \LaTeX.  If
someone finds it interesting to look how it was made, source code is available
on GitHub: \url{https://github.com/krzychu/c25k}. If you find any errors,
please report them there.


\newpage

%---------------------------------------------------------------------
%                             WEEK 1
%---------------------------------------------------------------------

\newcommand{\weekOneWorkout}{
  WALK & 05:00 \\
  \hline
  \multicolumn{2}{c}{\small{repeat $\Downarrow$ for 20:00 } } \\
  \hline
  RUN & 00:60 \\
  WALK & 00:90 
}

\workoutcard{lime}{ WEEK 1 DAY 1 }{\weekOneWorkout}
\workoutcard{lime}{ WEEK 1 DAY 2 }{\weekOneWorkout}
\workoutcard{lime}{ WEEK 1 DAY 3 }{\weekOneWorkout}

%---------------------------------------------------------------------
%                             WEEK 2
%---------------------------------------------------------------------
\workoutcard{olive}{ WEEK 2 DAY 1 }{\weekOneWorkout}
\workoutcard{olive}{ WEEK 2 DAY 2 }{\weekOneWorkout}
\workoutcard{olive}{ WEEK 2 DAY 3 }{\weekOneWorkout}

%---------------------------------------------------------------------
%                             WEEK 3
%---------------------------------------------------------------------

\newcommand{\weekThreeWorkout}{
  WALK & 05:00 \\
  \hline
  \multicolumn{2}{c}{\small{repeat $\Downarrow \times 2$} } \\
  \hline
  RUN & 00:90 \\
  WALK & 00:90 \\
  RUN & 03:00 \\
  WALK & 03:00
}

\workoutcard{brown}{ WEEK 3 DAY 1 }{\weekThreeWorkout}
\workoutcard{brown}{ WEEK 3 DAY 2 }{\weekThreeWorkout}
\workoutcard{brown}{ WEEK 3 DAY 3 }{\weekThreeWorkout}

%---------------------------------------------------------------------
%                             WEEK 4
%---------------------------------------------------------------------

\newcommand{\weekFourWorkout}{
  WALK & 05:00 \\
  RUN & 03:00 \\
  WALK & 00:90 \\
  RUN & 05:00 \\
  WALK & 02:30 \\
  RUN & 03:00 \\
  WALK & 00:90 \\
  RUN & 05:00 \\
}

\workoutcard{yellow}{ WEEK 4 DAY 1 }{\weekFourWorkout}
\workoutcard{yellow}{ WEEK 4 DAY 2 }{\weekFourWorkout}
\workoutcard{yellow}{ WEEK 4 DAY 3 }{\weekFourWorkout}


%---------------------------------------------------------------------
%                             WEEK 5
%---------------------------------------------------------------------

\workoutcard{orange}{ WEEK 5 DAY 1 }{
  WALK & 05:00 \\
  RUN & 05:00 \\
  WALK & 03:00 \\
  RUN & 05:00 \\
  WALK & 03:00 \\
  RUN & 05:00 
}
\workoutcard{orange}{ WEEK 5 DAY 2 }{
  WALK & 05:00 \\
  RUN & 08:00 \\
  WALK & 03:00 \\
  RUN & 08:00 
}
\workoutcard{orange}{ WEEK 5 DAY 3 }{
  WALK & 05:00 \\
  RUN & 20:00 
}
%---------------------------------------------------------------------
%                             WEEK 6
%---------------------------------------------------------------------

\workoutcard{red}{ WEEK 6 DAY 1 }{
  WALK & 05:00 \\
  RUN & 05:00 \\
  WALK & 03:00 \\
  RUN & 08:00 \\
  WALK & 03:00 \\
  RUN & 05:00 
}
\workoutcard{red}{ WEEK 6 DAY 2 }{
  WALK & 05:00 \\
  RUN & 10:00 \\
  WALK & 03:00 \\
  RUN & 10:00 
}
\workoutcard{red}{ WEEK 6 DAY 3 }{
  WALK & 05:00 \\
  RUN & 22:00 
}


%---------------------------------------------------------------------
%                             WEEK 7
%---------------------------------------------------------------------

\newcommand{\boringWeek}[3]{
\workoutcard{#2}{ WEEK #3 DAY 1 }{WALK & 05:00 \\ RUN & #1}
\workoutcard{#2}{ WEEK #3 DAY 2 }{WALK & 05:00 \\ RUN & #1}
\workoutcard{#2}{ WEEK #3 DAY 3 }{WALK & 05:00 \\ RUN & #1}
}

\boringWeek{25:00}{cyan}{7}

%---------------------------------------------------------------------
%                             WEEK 8
%---------------------------------------------------------------------
\boringWeek{28:00}{teal}{8}

%---------------------------------------------------------------------
%                             WEEK 9
%---------------------------------------------------------------------
\boringWeek{30:00}{magenta}{9}






\end{document}
